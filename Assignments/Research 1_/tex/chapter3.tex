% !TeX root=../main.tex

\chapter*{تحقیق اول، ماتریس پاد متقارن}

\section*{مقدمه}
یک ماتریس پادمتقارن $A$ یک ماتریس مربعی است که شرط زیر را برقرار می‌کند:
\begin{equation}
	A^T = -A.
\end{equation}
این ماتریس‌ها کاربردهای متعددی در فیزیک، مهندسی و گرافیک کامپیوتری دارند.

\section*{خواص ماتریس‌های پادمتقارن}

\subsection*{دترمینان}
دترمینان یک ماتریس پادمتقارن با مرتبه فرد همیشه صفر است:
\begin{equation}
	\det(A) = 0, \quad \text{اگر } A \text{ دارای مرتبه } (2n+1) \times (2n+1) \text{ باشد}.
\end{equation}
این نتیجه از خاصیت $\det(A) = \det(A^T) = \det(-A) = (-1)^n \det(A)$ حاصل می‌شود.

\subsection*{مقادیر ویژه}
مقادیر ویژه یک ماتریس پادمتقارن یا صفر هستند یا به‌صورت اعداد موهومی محض:
\begin{equation}
	\lambda = i \mu, \quad \mu \in \mathbb{R}.
\end{equation}
این خاصیت از معادله مشخصه ماتریس و تبدیلات مشابهت آن نتیجه می‌شود.

\subsection*{اثر}
از آنجا که عناصر قطری یک ماتریس پادمتقارن همواره صفر هستند، اثر آن نیز صفر خواهد بود:
\begin{equation}
	\text{tr}(A) = \sum_{i} A_{ii} = 0.
\end{equation}

\subsection*{رتبه}
رتبه یک ماتریس پادمتقارن همواره یک عدد زوج است. این ویژگی از تجزیه ماتریس به فرم‌های متعارف آن ناشی می‌شود.

\section*{عملیات روی ماتریس‌های پادمتقارن}

\subsection*{جمع و ضرب عددی}
اگر $A$ و $B$ ماتریس‌های پادمتقارن باشند، مجموع آن‌ها نیز پادمتقارن خواهد بود:
\begin{equation}
	(A + B)^T = A^T + B^T = -A - B = -(A+B).
\end{equation}
به همین ترتیب، برای هر عدد حقیقی $k$ داریم:
\begin{equation}
	(kA)^T = kA^T = k(-A) = -kA.
\end{equation}

\subsection*{ضرب ماتریسی}
اگر $A$ و $B$ ماتریس‌های پادمتقارن باشند، حاصل‌ضرب آن‌ها در صورتی که جابه‌جایی‌پذیر باشند، پادمتقارن خواهد بود:
\begin{equation}
	AB = -BA \Rightarrow AB \text{ پادمتقارن است}.
\end{equation}

\section*{کاربردهای ماتریس‌های پادمتقارن}

\subsection*{مکانیک}
ماتریس‌های پادمتقارن برای نمایش تکانه زاویه‌ای و تسهیل محاسبات در دینامیک چرخشی به کار می‌روند.

\subsection*{الکترومغناطیس}
تنسور میدان الکترومغناطیسی یک ماتریس پادمتقارن است که میدان‌های الکتریکی و مغناطیسی را ترکیب می‌کند.

\subsection*{گرافیک کامپیوتری}
در تبدیل‌های دورانی و محاسبات ضرب خارجی در رندر سه‌بعدی استفاده می‌شوند.

\subsection*{کنترل}
ماتریس‌های پادمتقارن برای توصیف تقارن‌ها در سامانه‌های هامیلتونی و قوانین پایستگی به کار می‌روند.

\subsection*{مکانیک کوانتومی}
در مطالعه عملگرهای اسپین و تبدیلات بنیادی ظاهر می‌شوند.

\section*{ارتباط با ماتریس‌های متقارن}
هر ماتریس مربعی $B$ را می‌توان به دو مؤلفه متقارن و پادمتقارن تجزیه کرد:
\begin{equation}
	B = \frac{1}{2}(B + B^T) + \frac{1}{2}(B - B^T).
\end{equation}
این تجزیه تحلیل و دسته‌بندی ماتریس‌ها را ساده‌تر می‌کند.

\section*{نتیجه‌گیری}
ماتریس‌های پادمتقارن نقش مهمی در کاربردهای ریاضی و مهندسی دارند. ویژگی‌های آن‌ها از جمله قیود روی دترمینان، مقادیر ویژه، و تجزیه ساختاری، آن‌ها را به ابزارهای اساسی در علوم نظری و کاربردی تبدیل کرده است.
