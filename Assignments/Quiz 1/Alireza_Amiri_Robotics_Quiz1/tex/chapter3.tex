% !TeX root=../main.tex

\chapter*{پاسخ کوییز اول}
\section*{پاسخ سوال}
در محاسبه ماتریس ژاکوبین، استفاده از روش مشتق‌گیری مستقیم از ماتریس مکان نقطه انتهایی (ماتریس \(P\)) به دلیل پیچیدگی‌های ذاتی و نیاز به محاسبات سنگین، به‌ویژه برای ربات‌هایی با تعداد لینک‌ها و مفاصل زیاد، معمولاً مناسب نیست. این ماتریس به‌طور مستقیم حاوی حجم عظیمی از اطلاعات است که مشتق‌گیری از آن به یک‌باره، فرآیندی پیچیده، زمان‌بر و مستلزم استفاده از بسته‌های محاسبات نمادین (Symbolic) است. علاوه بر پیچیدگی استفاده از این بسته ها، محاسبات آنها نیز به دلیل پیچیدگی روابط به سرعت افزایش یافته و استفاده از آنها در سیستم‌های زمان‌واقعی را غیرعملی می‌کند.

در مقابل، روش بازگشتی با استفاده از ضرب ماتریس‌های کوچک و محاسبه گام‌به‌گام ستون‌های ماتریس ژاکوبین، این مشکل را برطرف می‌کند. در این روش، به جای مواجهه با نتیجه نهایی به‌صورت یک‌باره، هر ستون از ماتریس ژاکوبین به‌صورت تدریجی و با استفاده از ماتریس‌های چرخش و بردارهای انتقال محاسبه می‌شود. این فرآیند به دلیل حجم کمتر محاسبات در هر گام، نه تنها ساده‌تر است بلکه بسیار سریع‌تر نیز انجام می‌شود. همچنین، ضرب ماتریس‌ها (مانند ضرب، ترانهاده، و سایر عملیات ماتریسی) به مراتب کارآمدتر از مشتق‌گیری نمادین است و نیاز به استفاده از روش‌های پیچیده و سنگین را از بین می‌برد. 

نکته مهم دیگر این است که تابع \texttt{jacobian} در نرم‌افزار MATLAB برای محاسبه ماتریس ژاکوبین نیز از مشتق‌گیری استفاده می‌کند. این تابع برای هر عنصر ماتریس، مشتق مربوط به متغیرهای ورودی را محاسبه می‌کند. بنابراین، همان‌طور که پیش‌تر اشاره شد، این روش مشتق‌گیری به دلیل حجم بالای محاسبات و غیرعملی بودن آن در سیستم‌های زمان‌واقعی، در کاربردهای عملی ترجیح داده نمی‌شود.

در نهایت، روش بازگشتی نه تنها زمان محاسبات را کاهش می‌دهد، بلکه به دلیل تکیه بر ساختار هندسی و سینماتیکی ربات، به روشی پایدار و قابل اعتماد برای کنترل و تحلیل حرکات ربات در سیستم‌های چندلینکی تبدیل شده است. این روش همچنین با افزایش تعداد لینک‌ها و مفاصل، پیچیدگی محاسبات را به صورت خطی افزایش می‌دهد و از رشد نمایی یا غیرقابل کنترل جلوگیری می‌کند، که این موضوع در سیستم‌های زمان‌واقعی اهمیت ویژه‌ای دارد.
