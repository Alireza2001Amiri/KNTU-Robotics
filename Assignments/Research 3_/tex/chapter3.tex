% !TeX root=../main.tex

\chapter*{تحقیق سوم، بسط رابطه نیروی نیوتن}

\section*{مقدمه}
معادلات اویلر-کین یک تعمیم مهم از مکانیک کلاسیک هستند که برای توصیف دقیق‌تر حرکت جسم صلب توسعه یافته‌اند. این معادلات شامل توصیف همزمان حرکت انتقالی و چرخشی اجسام بوده و بر پایه اصول نیوتن و اویلر بنا شده‌اند.

\section*{قانون دوم نیوتن و نقص آن}
قانون دوم نیوتن بیان می‌کند:
\begin{equation}
	F = ma
\end{equation}
این معادله تنها حرکت انتقالی را توصیف می‌کند و نیروهای داخلی، قیود و نیروهای اینرسی در دستگاه‌های غیر لخت را در نظر نمی‌گیرد.

\section*{بسط معادلات اویلر}
اویلر این معادلات را گسترش داد تا حرکت چرخشی را نیز شامل شود:
\begin{equation}
	M = I \alpha
\end{equation}
که در آن:
\begin{itemize}
	\item $M$ گشتاور اعمالی است،
	\item $I$ تانسور ممان اینرسی،
	\item $\alpha$ شتاب زاویه‌ای.
\end{itemize}

\section*{فرمول‌بندی کلی معادلات اویلر-کین}
بعدها کین معادلات نیوتن-اویلر را به فرم کلی‌تری بسط داد که برای سیستم‌های چندجسمی کاربرد دارد:
\begin{equation}
	M \ddot{q} + C(q, \dot{q}) \dot{q} + G(q) = Q
\end{equation}
که در آن:
\begin{itemize}
	\item $q$ مختصات تعمیم‌یافته،
	\item $M$ ماتریس جرم تعمیم‌یافته،
	\item $C(q, \dot{q})$ اثرات کوریولیس و گریز از مرکز،
	\item $G(q)$ نیروهای گرانشی،
	\item $Q$ نیروهای تعمیم‌یافته خارجی.
\end{itemize}

\section*{کاربردها}
معادلات اویلر-کین در زمینه‌های مختلفی از جمله:
\begin{itemize}
	\item طراحی سیستم‌های رباتیکی،
	\item دینامیک وسایل نقلیه،
	\item تحلیل سیستم‌های چندجسمی،
	\item شبیه‌سازی فیزیکی در مهندسی مکانیک و هوافضا.
\end{itemize}

\section*{نتیجه‌گیری}
معادلات اویلر-کین با در نظر گرفتن نیروهای اینرسی، قیود و حرکت چرخشی، یک ابزار قدرتمند برای تحلیل سیستم‌های مکانیکی پیچیده فراهم می‌کنند.
