% !TeX root=../main.tex

\chapter*{تحقیق چهارم، جمله ی کوریولیس}

\section*{مقدمه}
کوریولیس یکی از مهم‌ترین مفاهیم در دینامیک سیستم‌های دوار است که تاثیر بسزایی در توصیف حرکت اجسام در چارچوب مرجع چرخان دارد. این اثر که برای اولین بار توسط گاسپار-گوستاو دو کوریولیس در سال ۱۸۳۵ معرفی شد، منجر به تغییراتی در معادلات حرکت، به‌ویژه در مکانیک لاگرانژی می‌شود.

\section*{فرمول‌بندی ریاضی ترم کوریولیس}
نیروی کوریولیس یک نیروی مجازی است که در چارچوب‌های چرخان ظاهر می‌شود و به‌صورت زیر تعریف می‌شود:
\begin{equation}
	\mathbf{a}_C = -2\mathbf{v} \times \boldsymbol{\omega}
\end{equation}
که در آن:
\begin{itemize}
	\item \( \mathbf{v} \) سرعت جسم در چارچوب مرجع چرخان است.
	\item \( \boldsymbol{\omega} \) بردار سرعت زاویه‌ای چارچوب مرجع است.
\end{itemize}

\section*{نقش ترم کوریولیس در مکانیک لاگرانژی}
در فرمول‌بندی لاگرانژی، معادلات حرکت از اصل کمترین کنش استخراج می‌شوند و تأثیر ترم کوریولیس را در انرژی جنبشی سیستم نشان می‌دهند. معادله اویلر-لاگرانژ به‌صورت زیر بیان می‌شود:
\begin{equation}
	\frac{d}{dt} \left( \frac{\partial L}{\partial \dot{q}^i} \right) - \frac{\partial L}{\partial q^i} = 0
\end{equation}
که در آن \( L = T - V \) لاگرانژی سیستم است. در حضور چرخش، انرژی جنبشی شامل ترم‌های وابسته به سرعت و ترم کوریولیس می‌شود.

\section*{تأثیر بر دینامیک سیستم‌ها}
ترم کوریولیس نقش کلیدی در توصیف حرکت در سیستم‌های چرخان مانند:
\begin{itemize}
	\item رباتیک و مکانیزم‌های چرخان
	\item دینامیک جو و تشکیل گردبادها
	\item جریان‌های اقیانوسی و تغییرات آب و هوایی
\end{itemize}
دارد و در بسیاری از کاربردهای مهندسی و علمی نقش تعیین‌کننده‌ای ایفا می‌کند.

\section*{نتیجه‌گیری}
ترم کوریولیس، یک مولفه اساسی در دینامیک سیستم‌های چرخان است که حذف آن منجر به تخمین‌های نادرست در مدل‌سازی حرکت اجسام می‌شود. درک صحیح این پدیده به بهبود پیش‌بینی‌های علمی در زمینه‌های متعددی از جمله هواشناسی، ناوبری و فیزیک سیستم‌های دوار کمک می‌کند.
