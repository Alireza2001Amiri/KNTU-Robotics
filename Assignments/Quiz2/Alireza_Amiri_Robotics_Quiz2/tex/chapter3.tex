% !TeX root=../main.tex

\chapter*{پاسخ کوییز دوم}

\section*{مقدمه}
رباتیک به عنوان شاخه‌ای از علم و مهندسی، به توسعه و پیاده‌سازی ماشین‌هایی می‌پردازد که قادر به انجام وظایف انسانی هستند. این وظایف شامل جابجایی اشیا، شناسایی محیط از طریق حسگرها، و به کارگیری هوش مصنوعی برای تصمیم‌گیری است. در این دوره، تمرکز ما بر روی بخش مکانیکی و عملکردی ربات‌ها بوده و به موضوعات مرتبط با هوش مصنوعی کمتر پرداخته‌ایم.

ما با انواع ربات‌ها آشنا شدیم و دریافتیم که آن‌ها در دسته‌بندی‌های مختلفی مانند سریال، موازی و غیره قرار می‌گیرند. در این دوره، تمرکز اصلی بر روی ربات‌های سریال است که در آن هر لینک به یک مفصل متصل است. مفاصل اصلی شامل مفاصل انتقالی (پرزماتیک) و مفاصل چرخشی (روولوشن) هستند و دیگر انواع مفاصل نیز معمولاً ترکیبی از این دو نوع می‌باشند.

کاربردهای متنوع رباتیک شامل صنایع مختلفی مانند فضا، پزشکی، صنعت، هواپیماهای بدون سرنشین، ربات‌های انسان‌نما، و ربات‌های اکتشافی است که در حوزه‌های گوناگون مانند توانبخشی و کشف مناطق جدید به کار گرفته می‌شوند.

\section*{حرکت و توصیف موقعیت و جهت‌گیری ربات}
برای توصیف حرکت و موقعیت ربات، ابتدا مفهوم سیستم‌های مختصات معرفی شد. این سیستم‌ها باید ارتونورمال باشند و شامل سیستم‌های مختصات ثابت و متحرک می‌باشند. موقعیت یک نقطه از شیء نسبت به سیستم مختصات ثابت توصیف می‌شود، در حالی که جهت‌گیری به کل شیء مربوط است و نسبت به سیستم مختصات ثابت اندازه‌گیری می‌شود. ترکیب این دو جنبه، شش پارامتر را شامل می‌شود: سه پارامتر موقعیتی (محورهای $x$، $y$، $z$) و سه پارامتر چرخشی (معمولاً $\alpha$، $\beta$، $\gamma$). 

برای انتقال بین سیستم مختصات ثابت و متحرک، از ماتریس‌های چرخش استفاده شد. ماتریس‌های چرخش ویژگی‌هایی چون ارتونورمال بودن، برابری ترانهاده با معکوس و دترمینان برابر یک دارند. روش‌های مختلفی برای توصیف چرخش بررسی شد، شامل ماتریس‌های چرخش اویلر، چرخش معادل (که چرخش حول یک محور واحد در فضا را توصیف می‌کند)، و کواترنیون‌ها.

در ادامه، ماتریس تبدیل همگن معرفی شد که ترکیب ماتریس چرخش و ماتریس انتقال است و موقعیت و جهت‌گیری شیء را به طور یکپارچه توصیف می‌کند. همچنین، ویژگی‌هایی مانند معکوس ماتریس تبدیل بررسی شد و روش‌های کارآمدتری برای محاسبه آن ارائه گردید. قوانین زنجیره‌ای در ماتریس‌های تبدیل نشان دادند که برای یافتن موقعیت و جهت‌گیری نهایی شیء، می‌توان این ماتریس‌ها را به ترتیب ضرب کرد.

یکی از موضوعات اصلی این فصل، معرفی روش پیچش بود. این روش حرکت شیء را به صورت چرخش حول یک محور (محور پیچش) توصیف می‌کند. در این روش، هشت پارامتر شامل $\theta$ (مقدار چرخش)، $d$ (انتقال محور)، و دو بردار (بردار پیچش و برداری که مبدا را به محور پیچش متصل می‌کند) معرفی شدند.

در بخش پایانی، سرعت خطی و زاویه‌ای ربات محاسبه شد. مشتقات ماتریس‌های چرخش و تأثیر آن‌ها بر سرعت زاویه‌ای مورد بررسی قرار گرفت و مشخص شد که مشتق حرکت هر مفصل برابر با سرعت زاویه‌ای نهایی ربات نیست. همچنین، از روش‌هایی چون کواترنیون‌ها و اویلر برای محاسبات استفاده شد و سرعت خطی ربات نیز با استفاده از ماتریس چرخش به دست آمد. 

در پایان، مختصات پیچشی معرفی شد که مبنای نمایش جاکوبین است. تأثیر مفاصل پرزماتیک و چرخشی بر بردار پیچش بررسی شد و نشان داده شد که این بردارها ستون‌های ماتریس جاکوبین را تشکیل می‌دهند. علاوه بر این، کدنویسی در متلب و استفاده از لایو اسکریپت‌ها برای انجام این محاسبات توسعه داده شد و به عنوان ابزاری ارزشمند در این دوره مورد استفاده قرار گرفت.

\section*{سینماتیک: توصیف موقعیت و فضای کاری ربات}
سینماتیک به توصیف موقعیت، جهت‌گیری، و مقادیر زوایای مفاصل یک ربات می‌پردازد. در این فصل، سینماتیک مستقیم و معکوس معرفی شدند:

\subsection*{سینماتیک مستقیم}
سینماتیک مستقیم رابطه‌ای از فضای مفصلی به فضای کاری فراهم می‌کند، به طوری که با دانستن زوایای مفاصل و پارامترهای ربات، موقعیت و جهت‌گیری نهایی اندافکتور محاسبه می‌شود. روش‌های مختلفی برای توصیف سینماتیک مستقیم بررسی شدند:
\begin{itemize}
	\item \textbf{روش هندسی:} با استفاده از اصول هندسه و مثلثات، موقعیت اندافکتور محاسبه شد. این روش برای ربات‌های ساده مناسب است اما برای سیستم‌های پیچیده با درجات آزادی بالا دشوار می‌شود.
	\item \textbf{پارامترهای دنویت-هارتنبرگ (DH):} این روش با تعریف پارامترهای DH (طول لینک، زاویه پیچش، شیب لینک، و زاویه مفصل) و ساخت ماتریس تبدیل، روشی سیستماتیک برای محاسبات ارائه می‌دهد. زنجیره‌ی ماتریس‌های تبدیل امکان محاسبه موقعیت و جهت‌گیری اندافکتور را برای ربات‌های پیچیده نیز فراهم می‌کند. پیاده‌سازی این روش در MATLAB و استفاده از توابع موجود در آن بررسی شد.
	\item \textbf{روش پیچشی (Screw):} این روش که بر مبنای فیزیک حرکت ربات توسعه یافته است، با تعریف محور پیچش و پارامترهای پیچشی، ماتریس تبدیل را محاسبه می‌کند. این روش پیچیدگی محاسباتی بیشتری دارد اما درک بهتری از حرکت سیستم ارائه می‌دهد. در این بخش، محاسبه بردار پیچش، محور پیچش، و ماتریس تبدیل بررسی شد.
\end{itemize}

\subsection*{سینماتیک معکوس}
سینماتیک معکوس محاسبه زوایای مفاصل را بر اساس موقعیت و جهت‌گیری اندافکتور انجام می‌دهد. این مسئله برای کنترل ربات اهمیت بسیاری دارد. در این بخش:
\begin{itemize}
	\item حل هندسی سینماتیک معکوس برای ربات‌های ساده انجام شد.
	\item محدودیت‌های سینماتیک معکوس، مانند وجود چندین جواب ممکن یا عدم وجود جواب در برخی موقعیت‌ها بررسی شدند.
\end{itemize}

\subsection*{فضای کاری ربات}
فضای کاری مجموعه تمام موقعیت‌ها و جهت‌گیری‌هایی است که اندافکتور می‌تواند به آن‌ها دست یابد. در این بخش:
\begin{itemize}
	\item تحلیل فضای کاری با بررسی محدودیت‌های فیزیکی ربات انجام شد.
	\item موقعیت‌هایی که خارج از فضای کاری هستند و دسترسی به آن‌ها امکان‌پذیر نیست، شناسایی شدند.
\end{itemize}

\subsection*{پیاده‌سازی MATLAB و تمرینات عملی}
در طول این فصل، تمرینات متعددی برای پیاده‌سازی سینماتیک مستقیم و معکوس در MATLAB انجام شد. این تمرینات شامل:
\begin{itemize}
	\item پیاده‌سازی مثال‌های کتاب با استفاده از پارامترهای DH.
	\item بررسی روش پیچشی و محاسبات مربوط به آن.
	\item تحلیل فضای کاری برای ربات‌های با درجات آزادی مختلف (2، 6، یا بیشتر).
\end{itemize}
پیاده‌سازی این مفاهیم در MATLAB ابزار ارزشمندی برای حل مسائل پیچیده و شبیه‌سازی ربات‌ها فراهم کرد.

\section*{ماتریس ژاکوبی: تحلیل ویژگی‌های حرکت ربات}
در این فصل، ماتریس ژاکوبی به عنوان ابزاری کلیدی برای تحلیل جنبه‌های مختلف حرکت ربات معرفی شد. این تحلیل نه‌تنها موقعیت و جهت‌گیری، بلکه سرعت‌های خطی و زاویه‌ای، نیروها و گشتاورها، وضعیت‌های تکین، مهارت و ایزوتروپی ربات را نیز در بر می‌گیرد.

\subsection*{تعریف ماتریس ژاکوبی}
ماتریس ژاکوبی به صورت مشتق ماتریس سینماتیک مستقیم نسبت به متغیرهای مفصلی تعریف می‌شود. این ماتریس:
\begin{itemize}
	\item شامل اطلاعات مربوط به سرعت خطی و زاویه‌ای است.
	\item امکان بررسی رابطه بین فضای کاری و فضای مفصلی را فراهم می‌کند.
\end{itemize}

\subsection*{روش‌های محاسبه ماتریس ژاکوبی}
دو روش اصلی برای محاسبه ماتریس ژاکوبی بررسی شد:
\begin{itemize}
	\item \textbf{روش مشتق مستقیم:} محاسبه مشتقات جزئی ماتریس سینماتیک مستقیم نسبت به متغیرهای مفصلی. این روش برای ربات‌های با درجات آزادی بالا بسیار پیچیده است.
	\item \textbf{روش عمومی (بازگشتی):} محاسبه هر ستون ماتریس ژاکوبی به صورت تکراری. این روش شباهت زیادی به روش پیچشی دارد و پیچیدگی محاسباتی را کاهش می‌دهد.
\end{itemize}

\subsection*{کاربردهای ماتریس ژاکوبی}
\begin{itemize}
	\item \textbf{محاسبه نیرو و گشتاور:} با استفاده از ماتریس ژاکوبی و ترانهاده آن، رابطه بین نیروها و گشتاورهای اعمال‌شده در فضای کاری و فضای مفصلی بررسی شد. این رابطه در طراحی حلقه‌های کنترلی بسیار کاربردی است.
	\item \textbf{وضعیت‌های تکین:} وضعیت‌های تکین زمانی رخ می‌دهند که دترمینان ماتریس ژاکوبی صفر شود. این وضعیت‌ها معمولاً در لبه فضای کاری ظاهر می‌شوند و ویژگی‌های زیر دارند:
	\begin{itemize}
		\item سرعت پایین در برخی محورها.
		\item امکان اعمال نیروهای بسیار بزرگ.
	\end{itemize}
	\item \textbf{مهارت و ایزوتروپی:} مهارت ربات در مناطق مختلف فضای کاری متفاوت است. برای بررسی این موضوع، ماتریس حاصل‌ضرب ژاکوبی در ترانهاده آن تحلیل شد. ایزوتروپی حالتی است که ربات در تمام محورها و مناطق فضای کاری عملکردی یکسان داشته باشد.
\end{itemize}

\subsection*{شاخص‌های مهارت}
برای ارزیابی مهارت ربات، معیارهایی مانند مقادیر ویژه ماتریس ژاکوبی و شاخص ایزوتروپی تعریف و تحلیل شدند.

\subsection*{تمرینات عملی}
در این فصل، تمرینات متعددی با MATLAB انجام شد که شامل:
\begin{itemize}
	\item محاسبه ماتریس ژاکوبی برای ربات‌های با درجات آزادی مختلف.
	\item تحلیل وضعیت‌های تکین و محاسبه دترمینان ژاکوبی.
	\item بررسی مهارت و ایزوتروپی در فضای کاری.
\end{itemize}
این تمرینات، دانش عملی لازم برای استفاده از ماتریس ژاکوبی در طراحی و تحلیل سیستم‌های رباتیک را فراهم کردند.



