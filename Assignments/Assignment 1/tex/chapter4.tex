% !TeX root=../main.tex
\chapter{نتایج}
%\thispagestyle{empty} 
\label{chap:results}
\section{مقدمه} 
یکی از فصل‌های مهم  \پ  این فصل است که در آن به ارائهٔ داده‌ها، نتایج، تحلیل و تفسیر اولیهٔ آنها پرداخته می‌شود. در ارائهٔ نتایج تا حد امکان، ترکیبی از نمودارها و جداول استفاده شود، چراکه به کمک آن‌ها بهتر می‌توان نتایج کار را نمایش داد و نسبت به کار سایرین آن را متمایز کرد. با توجه به حجم و ماهیت تحقیق و با صلاحدید استاد راهنما، این فصل می‌تواند تحت عنوانی دیگر بیاید. در این فصل باید به سوالات تحقیق که در پیش‌تر در مقدمه و مروری بر ادبیات موضوع بیان شده است، بنابر یافته‌های محقق در روند انجام کار، پاسخ داده شود. اگر تحقیق دارای آزمون فرض باشد، پذیرش یا عدم پذیرش فرضیه‌ها در این فصل گزارش می‌شود. این فصل باید حدود 20 صفحه باشد.

\section{محتوا}
در این بخش به سوالات تحقیق، بر اساس داده‌ها و یافته‌های محقق، پاسخ داده می‌شود. داده‌ها با فرمت مناسبی ارائه می‌شوند؛ مدل (ها) اجرا شده و نتیجه آن مشخص می‌شود.

\section{اعتبارسنجی}
از طریق مقایسهٔ نتایج با نتایج کارهای دیگران، استفاده از روش‌های تحلیل پایائی
\lr{(reliability)}
و اعتبار
\lr{(validity)}،
نظرگیری از خبرگان
\lr{(expert judgment or feedback)}
و یا
\lr{triangulation}
انجام می‌شود.
