% !TeX root=../main.tex
% در این فایل، عنوان پایان‌نامه، مشخصات خود و چکیده پایان‌نامه را به انگلیسی، وارد کنید.

%%%%%%%%%%%%%%%%%%%%%%%%%%%%%%%%%%%%
\latinuniversity{K. N. Toosi University of Technology}
\latincollege{...}
\latinfaculty{Faculty of Electrical Engineering}
\latindepartment{Mechatronics Group}
\latinsubject{Mechatronics Engineering}
%\latinfield{field}
\latintitle{Analyzing Design and Implementation methods of Magnetically Levitated Planar Motors}
\firstlatinsupervisor{Mahdi Aliyari Shooredeli}
\secondlatinsupervisor{Esmaeil Najafi}
%\firstlatinadvisor{First Advisor}
%\secondlatinadvisor{Second Advisor}
\latinname{Alireza}
\latinsurname{Amiri}
\latinthesisdate{Summer 2024}
\latinkeywords{Magnetic levitation, Planar motors, Halbach array, Model Predicting Control, Contactless operation}
\en-abstract{
Magnetic levitation planar motors (MLPM) offer precise, contactless motion, making them ideal for applications requiring high-accuracy positioning, such as in manufacturing, automation, and robotics. However, designing and optimizing MLPM systems involves overcoming challenges related to system architecture, magnet configuration, control strategies, and modeling techniques, all of which significantly impact the performance and efficiency of these devices.
This study examines four key aspects of MLPM systems. First, the architecture of the system is analyzed, with a focus on the choice between moving-coil and fixed-coil designs. The findings suggest that fixed-coil designs, where the magnets are placed on the moving component, reduce physical constraints such as electrical connections and improve cooling efficiency, making them more suitable for real-world applications. Second, the design of permanent magnets is explored, specifically comparing disc magnets with Halbach arrays. Halbach arrays, especially in two-dimensional configurations, are found to provide stronger, more concentrated magnetic fields with greater control precision, outperforming traditional magnet designs.
In the third aspect, control strategies are evaluated, with classical PID controllers being compared to more advanced techniques like Model Predictive Control (MPC) and AI-based methods such as Gated Recurrent Units (GRU). While PID controllers are effective for basic applications, advanced control techniques leveraging system dynamics and machine learning demonstrate improved stability and reduced error in controlling MLPM systems.
Finally, the study explores modeling approaches, contrasting analytical methods with numerical techniques like finite element modeling (FEM). The analysis reveals that while analytical models provide a foundational understanding, numerical simulations, particularly FEM, offer greater accuracy and flexibility in designing and validating complex MLPM systems.
In conclusion, this research highlights the advantages of using fixed-coil architecture, Halbach magnet arrays, advanced control strategies, and numerical modeling techniques for optimizing MLPM performance, paving the way for their broader application in precision-driven industries.
}
