% !TeX root=../main.tex

\chapter*{پاسخ سوالات سری سوم}
\section*{پاسخ سوال سه}

برای محاسبه ی ماتریس ژاکوبین ربات هفت درجه آزادی خواسته شده در صورت سوال، ابتدا پارامتر های DH ربات را بر اساس پاسخ های به دست آمده در بخش های قبل وارد می کنیم. در اینجا، به منظور تسهیل در فرایند تعریف ماتریس های تبدیل DH، ابتدا پارامترها در ماتریسی تعریف شده و سپس، در یک حلقه ی for ماتریس های دوران مربوط به هر مفصل محاسبه و ذخیره می شوند. 
\[
\begin{pmatrix}
	0 & -\frac{\pi}{2} & a_1 & \theta_1 \\
	0 & \frac{\pi}{2} & 0 & \theta_2 \\
	0 & \frac{\pi}{2} & a_3 & \theta_3 \\
	0 & -\frac{\pi}{2} & 0 & \theta_4 \\
	0 & -\frac{\pi}{2} & a_5 & \theta_5 \\
	0 & \frac{\pi}{2} & 0 & \theta_6 \\
	0 & 0 & a_7 & \theta_7
\end{pmatrix}
\]

\subsection*{سینماتیک مستقیم بر اساس DH}
برای ذخیره ی ماتریس های دوران در هر مفصل نسبت به مختصات مرجع، این مقادیر در یک حلقه ی for مجزا محاسبه شده و در هر مرحله، در یک tensor به نام T-Total ذخیره می شود. به دلیل حجم بالای ماتریس های محاسبه شده برای این تبدیل ها، خروجی متلب در این بخش آورده نمی شود اما در کد متلب موجود می باشد.

در گام بعد، با استفاده از تابع PR-Link، ماتریس های دوران و انتقال از ماتریس های تبدیل به دست آمده از مراحل قبل جدا شده و در دو ماتریس P-Total-DH و R-Total-DH ذخیره می شوند.

در نهایت، برای مشخص کردن ماتریس دوران و انتقال نهایی به عنوان آخرین ماتریس در تنسور های بالا در دو ماتریس P-End-Effector-DH و R-End-Effector-DH استخراج می شوند.

\subsection*{سینماتیک مستقیم بر اساس پیچه}

مشابه آنچه که در بخش قبل انجام شد، در این قسمت ابتدا پارامتر های پیچه در یک ماتریس تعریف شده و سپس در یک حلقه ی for، ماتریس های تبدیل حاصل از این تبدیل ها محاسبه می شوند. 
\[
\begin{pmatrix}
	0 & 0 & 1 & 0 & 0 & 0 & \theta_1 & 0 \\
	0 & 1 & 0 & 0 & 0 & a_1 & \theta_2 & 0 \\
	0 & 0 & 1 & 0 & 0 & 0 & \theta_3 & 0 \\
	0 & -1 & 0 & 0 & 0 & a_1 + a_3 & \theta_4 & 0 \\
	0 & 0 & 1 & 0 & 0 & 0 & \theta_5 & 0 \\
	0 & 1 & 0 & 0 & 0 & a_1 + a_3 + a_5 & \theta_6 & 0 \\
	0 & 0 & 1 & 0 & 0 & 0 & \theta_7 & 0
\end{pmatrix}
\]

ماتریس های تبدیل به دست آمده از این روش نیز در متلب محاسبه و نمایش داده شده است.
پس از این، موقعیت و جهت گیری ربات در نقطه ی ابتدایی به صورت زیر تعیین می شود.
\[
\begin{pmatrix}
	1 & 0 & 0 & 0 \\
	0 & 1 & 0 & 0 \\
	0 & 0 & 1 & a_1 + a_3 + a_5 + a_7 \\
	0 & 0 & 0 & 1
\end{pmatrix}
\]
 با در اختیار داشتن ماتریس های تبدیل بین هر دو مفصل، می توان ماتریس های تبدیل هر مفصل را نسبت به ماتریس مرجع در یک حلقه ی بازگشتی محاسبه کرد و در نهایت، ماتریس های دوران و انتقال را از ماتریس های تبدیل به دست امده جدا کرد.
 
 در پایان این بخش، سینماتیک به دست آمده از دو روش DH و پیچه با یکدیگر مقایسه و اعتبارسنجی می شوند.
 
 \[
 \text{Error}_P =
 \begin{pmatrix}
 	0 \\
 	0 \\
 	0
 \end{pmatrix}
 \]
 
 \[
 \text{Error}_R =
 \begin{pmatrix}
 	0 & 0 & 0 \\
 	0 & 0 & 0 \\
 	0 & 0 & 0
 \end{pmatrix}
 \]
 
 مشاهده می شود که این دو روش نتایج یکسانی را به دست داده اند و می توانیم با اطمینان از این نتایج، ماتریس ژاکوبین را در بخش های بعد محاسبه می کنیم. با این حال، مشاهده می شود که تعدادی از ماتریس های دوران در میانه ی مسیر، با یکدیگر برابر نیستند.
 
\subsection*{ماتریس ژاکوبین به روش DH}

برای محاسبه ی ماتریس ژاکوبین، با استفاده از روابط 4.33، 4.34 و 4.35 ماتریس های p و z محاسبه می شوند. 
\[
r = \begin{pmatrix}
	a \cos(\theta) \\
	a \sin(\theta) \\
	d
\end{pmatrix}
\]


با در اختیار داشتن این ماتریس ها برای هر دوران با استفاده از دو روش ضرب خارجی و یا ماتریس پادمتقارن، هر یک از ستون های ماتریس ژاکوبین محاسبه می شوند.
\[
\text{skewMatrix} = 
\begin{pmatrix}
	0 & -\text{vec1}_3 & \text{vec1}_2 \\
	\text{vec1}_3 & 0 & -\text{vec1}_1 \\
	-\text{vec1}_2 & \text{vec1}_1 & 0
\end{pmatrix}
\]

در پایان این روش، ماتریس ژاکوبین مبتنی بر روش DH به دست می آید. مشاهده می شود که برای محاسبه ی ماتریس ژاکوبین با استفاده از روش فوق، $6.73$ ثانیه زمان نیاز است. در ادامه، با محاسبه ی این مقدار با روش پیچه، می توان مقایسه ای برای عملکرد دو روش بر اساس زمان اجرا داشت.

\subsection*{محاسبه ماتریس ژاکوبین به روش پیچه}
محاسبه ی ماتریس ژاکوبین به روش پیچه، شامل فرایند تکمیل کردن دو ماتریس s و so است که طبق روابط 4.47 و 4.48 باید تعیین شوند. همچنین، انجام این محاسبات باید ماتریس p مطابق با رابطه ی 4.49 مشخص شود و از آنجا که در ربات هفت درجه آزادی مورد استفاده در این کد، تمامی مقادیر d برابر با صفر هستند، تنها ماتریس هایی با درایه اول برابر با طول پیوند استفاده شده است. 
در نهایت، با استفاده از روش بازگشتی برای محاسبه ی بردارهای$ s_i$ و$ so_i$، ستون های ماتریس ژاکوبین مطابق رابطه ی 4.44 محاسبه می شود. لازم به ذکر است که در اینجا نیز، به جای استفاده از ضرب خارجی، از ماتریس پادمتقارن برای محاسبه ی این مقدار استفاده شده است.

لازم است توجه شود که با جایگذاری ماتریس های دوران به دست آمده از روش پیچه در این کد، قادر نخواهیم بود پاسخ های یکسانی را با روش DH به دست اوریم. بنابراین، در قسمتی از کد که لازم است از ماتریس های دوران استفاده کنیم، ماتریس های به دست امده از روش DH را مورد استفاده قرار داده ایم. زمان محاسبه ی ماتریس ژاکوبین به این روش برابر با $7.74$ ثانیه می باشد.
با در نظر داشتن این مورد، در نهایت می توانیم تطابق بین پاسخ های به دست آمده برای دو روش DH و پیچه را بررسی کنیم که در این صورت خواهیم داشت:
\[
\text{Error}_{jw} = \text{Jw}_{Screw} - \text{Jw}_{DH} = 
\begin{pmatrix}
	0 & 0 & 0 & 0 & 0 & 0 & 0 \\
	0 & 0 & 0 & 0 & 0 & 0 & 0 \\
	0 & 0 & 0 & 0 & 0 & 0 & 0
\end{pmatrix}
\]
\[
\text{Error}_{jv} = \text{Jv}_{Screw} - \text{Jv}_{DH} = 
\begin{pmatrix}
	0 & 0 & 0 & 0 & 0 & 0 & 0 \\
	0 & 0 & 0 & 0 & 0 & 0 & 0 \\
	0 & 0 & 0 & 0 & 0 & 0 & 0
\end{pmatrix}
\]

با این نتیجه گیری، مشاهده می شود که روش های ارائه شده برای محاسبه ی ماتریس ژاکوبین کارامد بوده و این ماتریس به درستی محاسبه شده است.

\textbf{\textit{به دلیل فضای زیادی که برای نمایش نتایج به دست آمده از این بخش مورد نیاز است و امکان چاپ آنها در فضای گزارش موجود نیست، خروجی های کد در گزارش آورده نشده اند. این نتایج در کد متلب مربوط به این تمرین ارائه شده است.}}












