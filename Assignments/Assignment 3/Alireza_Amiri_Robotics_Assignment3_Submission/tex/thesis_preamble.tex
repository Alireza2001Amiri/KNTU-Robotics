% ابتدای درج صفحات مختلف
%\coverPage
% بررسی حالت پیش‌نویس
\ifoptiondraft{}{% 
    \titlePage
    %\besmPage
% چنانچه مایل به چاپ صفحات «تقدیم»، «نیایش» و «سپاس‌گزاری» در خروجی نیستید، خط‌های زیر را با گذاشتن ٪  در ابتدای آنها غیرفعال کنید.
%    \taghdimPage
 %  \davaranPage
%%%%%%%%%%%%%%%%%%%%%%%%%%%
%    \esalatPage
%    \mojavezPage
%    \ghadrdaniPage
} % end ifoptiondraft
%\abstractPage
% شروع درج فهرست‌ها

%\cleardoublepage
\clearpage
\pagenumbering{harfi} % آ، ب، ...
%\tableofcontents \clearpage
% بررسی حالت پیش‌نویس برای بقیه فهرست‌ها
\ifoptiondraft{
 %   \listoftodos
}{%

% % Redefining the name of the list of figures to "فهرست شکل‌ها"
% \renewcommand{\listfigurename}{فهرست شکل‌ها}

% % Redefining the name of the list of tables to "فهرست جدول‌ها"
% \renewcommand{\listtablename}{فهرست جدول‌ها}

%\listoffigures \clearpage
%\listoftables  \clearpage
% فهرست الگوریتم‌ها: اگر نیازی به ایجاد فهرست الگوریتم‌ها ندارید دو خط زیر را کامنت کنید
%\addcontentsline{toc}{chapter}{\listalgorithmname}
%\listofalgorithms \clearpage
% فهرست برنامه‌ها: اگر نیازی به ایجاد فهرست برنامه‌ها ندارید دو خط زیر را کامنت کنید
%\addcontentsline{toc}{chapter}{\lstlistlistingname}
%\lstlistoflistings \clearpage
% فهرست اختصارات: اگر نیازی به ایجاد فهرستاختصارات ندارید خط زیر را کامنت کنید
%\printacronyms
} % end ifoptiondraft
